\documentclass[11pt,letterpaper]{report}

\usepackage{amsmath}
\usepackage{amsfonts}
\usepackage{amssymb}
\usepackage{graphicx}
\usepackage{tabularx}
\usepackage{algorithm}
\usepackage{algpseudocode} 
\usepackage{subcaption}
\usepackage[font=small]{caption}
\usepackage{url}
\usepackage[english]{babel}
\usepackage{setspace} 
\usepackage[margin=1in]{geometry}

\usepackage{cjhebrew}
%\usepackage{times}
%\usepackage{venturis2}

\doublespacing

\title{\textsc{Towards a Framework for DHT Distributed Computing}}
\author{\textsc{Andrew Benjamin Rosen}}
\date{}

\begin{document}
	
	

	\pagestyle{empty}

	\begin{center}
		%	\includegraphics[width=0.15\textwidth]{example-image-1x1}\par\vspace{1cm}
		{\scshape Towards a Framework for DHT Distributed Computing\par}
		\vspace{0.5cm}
		{by\par}
		\vspace{0.5cm}
		{\scshape Andrew Benjamin Rosen\par}
		\vspace{0.5cm}
		{Under the Direction of Dr.~Anu G.~Bourgeois, PhD \par}
		\vspace{0.5cm}
		{\scshape Abstract \par}
		
	\end{center}
	
	
		Distributed Hash Tables (DHTs) are protocols and frameworks used by peer-to-peer (P2P) systems.
		They are used as the organizational backbone for many P2P file-sharing systems due to their scalability, fault-tolerance, and load-balancing properties.
		These same properties are highly desirable in a distributed computing environment, especially one that wants to use heterogeneous components.
		
		We show that DHTs can be used not only as the framework to build a P2P file-sharing service, but as a P2P distributed computing platform.
		We propose creating a P2P distributed computing framework using distributed hash tables, based on our prototype system ChordReduce.
		This framework would make it simple and efficient for developers to create their own distributed computing applications.
		Unlike Hadoop and similar MapReduce frameworks, our framework can be used both in both the context of a datacenter or as part of a P2P computing platform.  
		This opens up new possibilities for building platforms to distributed computing problems.
		
		One advantage our system will have is an autonomous load-balancing mechanism.
		Nodes will be able to independently acquire work from other nodes in the network, rather than sitting idle.
		More powerful nodes in the network will be able use the mechanism to acquire more work, exploiting the heterogeneity of the network.
		
		By utilizing the load-balancing algorithm, a datacenter could easily leverage additional P2P resources at runtime on an as needed basis.
		Our framework will allow MapReduce-like or distributed machine learning platforms to be easily deployed in a greater variety of contexts.
		
		\vfill
		

		% Bottom of the page
		{\textsc{Index Words}: Distributed Hash Tables, P2P, Voronoi, Delaunay, Networking}
	
	
	\newpage 
	\begin{titlepage}
		
		\begin{center}
			%	\includegraphics[width=0.15\textwidth]{example-image-1x1}\par\vspace{1cm}
			{\scshape Towards a Framework for DHT Distributed Computing\par}
			\vspace{5cm}
			{by\par}
			\vspace{5cm}
			{\scshape Andrew Benjamin Rosen\par}
			
			\vfill
			A Dissertation Submitted in Partial Fulfillment of the Requirements for the Degree of\\
			Doctor of Philosophy in Computer Science\\
			in the College of Arts and Sciences\\
			Georgia State University\\
			2016
		\end{center}
	\end{titlepage}
	
	
	
	
	
	
	\null\vfill
	\begin{center}
		
		Copyright by \\
		Andrew Benjamin Rosen\\
		\begin{cjhebrew}
			.hnwK
		\end{cjhebrew}\\
		2016
	\end{center}
	\newpage
	


	\begin{center}
		{\scshape Towards a Framework for DHT Distributed Computing\par}
		\vspace{5cm}
		{by\par}
		\vspace{5cm}
		{\scshape Andrew Benjamin Rosen\par}
	\end{center}
	\vfill
	\hfill\begin{tabular}{rl}
		Committee Chair & Anu G. Bourgeois \\ 
		&  \\ 
		Committee & Robert Harrison  \\ 
		& Yingshu Li\\ 
		& Michael Stewart 
	\end{tabular} 
	
	
	\noindent
	Electronic Version Approved:\\
	\vspace{1cm}
	
	\noindent
	Office of Graduate Studies \\
	College of Arts and Sciences \\
	Georgia State University\\
	May 2016 
	
	\newpage
	\pagestyle{plain}
	\pagenumbering{roman}
	
	\chapter*{Dedication}
	I would like to take the time to thank Annie-Rae Rosen, without whom, I would not be who I am today.
	
	To my mother, who gave me a name that became a self-fulfilling prophecy.
	
	\chapter*{Acknowledgments}	
	There were some people who cared about what I did.  
	I'm not particularly sure why.
	
	
	\setcounter{tocdepth}{4}
	
	%\cleardoublepage
	%\addcontentsline{toc}{chapter}{\listtablename}
	%\addcontentsline{toc}{chapter}{\listfigurename}
	\tableofcontents
	\listoftables
	\listoffigures
	\newpage
	

	\clearpage
	\pagenumbering{arabic}
	\include{intro}
	\include{background}
	%\include{prevresearch}
	\chapter{ChordReduce}
\label{chapter:chordreduce}

Google's MapReduce \cite{mapreduce} paradigm has rapidly become an integral part in the world of data processing and is capable of efficiently executing numerous Big Data programming and data-reduction tasks.  By using MapReduce, a user can take a large problem, split it into small, equivalent tasks and send those tasks to other processors for computation.  The results are sent back to the user and combined into one answer.  Many popular platforms for MapReduce, such as Hadoop \cite{hadoop}, utilize a central source of coordination and organization to store and operate on data. The hierarchical structure of Hadoop results in a single point of failure at the node that concentrates the results and also requires a complicated scheme for handling node failures.

We developed a system, called ChordReduce, which employs a less hierarchical structure.  
It is a system that can scale, is fault tolerant, has a minimal amount of latency, and distributes tasks evenly.  ChordReduce leverages the underlying protocol from Chord \cite{chord} to distribute Map and Reduce tasks to nodes evenly, provide greater data redundancy, and guarantee a greater amount of fault tolerance. Rather than viewing Chord solely as a means for sharing files, we see it as a means for distributing work. This paper establishes the effectiveness of using Chord as a framework for distributed programming.   At the same time we avoid the architectural and file system constraints of systems like Hadoop.  


\section{Background}
ChordReduce takes its name from the two components it is built upon.  Chord \cite{chord} provides the backbone for the network and the file system, providing scalable routing, distributed storage, and fault-tolerance.   MapReduce runs on top of the Chord network and utilizes the underlying features of the distributed hash table.  This section provides background on Chord and MapReduce.


\subsection{Chord}
Chord \cite{Chord} is a P2P protocol for file sharing that uses a hash function to assign addresses to nodes and files for a ring overlay. The Chord protocol takes in some key and returns the identity (ID) of the node responsible for that key.  These keys are generated by hashing a value of the node, such as the IP address and port, or by hashing the filename of a file.  The hashing process creates a $m$-bit hash identifier.

The nodes are then arranged in a ring from the lowest hash-value to highest.  Chord takes the files and places each in the node that has the same hashed identifier as it.  If no such node exists, the node with the first identifier that follows this value is selected. Since the overlay is a circle, this assignment is computed in modulo $2^m$ space.  

The node responsible for the key $\kappa$ is called the $successor$ of $\kappa$, or $successor(\kappa)$.  For example, if there were some portion of the network with nodes 20, 25, and 27, node 25 would be responsible for the files with the keys (21,22,23,24,25). If node 25 were to decide to leave the network, its absence would be detected by node 27, who would then be responsible for all the keys node 25 was covering, in addition to its own keys. 
\begin{figure}
	\includegraphics[width=\linewidth]{figs/chordreal}
	\caption{A Chord ring with 16 nodes.  The bold lines are incoming edges.  Each node has a connection to its successor, as well as 4 fingers, some of which are duplicates.}
	\label{fig:chordreal}
\end{figure}


With this scheme, we can reliably find the node responsible for some key by asking the next node in the circle for the information, who would then pass the request through the circle until the successor was found.  We can then proceed to directly connect with the successor to retrieve the file.  This naive approach is largely inefficient, and is a simplification of the lookup process, but it is the basis of how Chord theoretically works.

To speed up the lookup time, each node builds and maintains a \emph{finger table}.  The \emph{finger table} contains the locations of up to $m$ other nodes in the ring.  The $i$th entry of node $n$'s \emph{finger table} corresponds to the node that is the $successor(n+2^{i-1})$ $mod$ $2^m$. Hash values are not perfectly distributed, it is possible to have duplicate entries in the \emph{finger table}. An example Chord network with fingers is shown in in Fig. \ref{chordreal}.


\begin{figure}
	\includegraphics[width=\linewidth]{figs/abcd1}
	\caption{Alice has incorrectly determined that Carol is her appropriate successor.  When Alice stabilizes, Carol will let her know about Bob.}
	\label{fig:abcd1}
\end{figure}


\begin{figure}
	\includegraphics[width=\linewidth]{figs/abcd2}
	\caption{After completing stabilize, Alice makes Bob her successor and notifies him. Bob then made Alice as his predecessor.}
	\label{fig:abcd2}
\end{figure}



When a node $n$ is told to find some key, $n$ looks to see if the key is between $n$ and $successor(n)$ and return $successor(n)$'s information to the requester. If not, it looks for the entry in the finger table for the closest preceding node $n'$ it knows and asks $n'$ to find the successor.  This allows each step to skip up to half the nodes in the network, giving a $\log_2(n)$ lookup time.  Because nodes can constantly join and leave the network, each entry in the table is periodically checked and updated during a finger maintenance period. 

To join the network, node $n$ first asks $n'$ to find $successor(n)$ for it.  Node $n$ uses the information to set his successor, but the other nodes in the ring will not acknowledge $n$'s presence yet.  Node $n$ relies on the stabilize routine to fully integrate into the ring.

The stabilize routine helps the network integrate new nodes and route around nodes who have left the network. Each node periodically checks to see who their successor's predecessor is.  In the case of a static network, this would be the checking node.  However, if the checking node gets back a different node, it looks at that returned node's hash value and changes its own successor if needed.  Regardless of whether the checking node changes its successor, that node then notifies the (possibly) new successor,  who then checks if he needs to change his predecessor based on this new information.  While complex, the stabilization process is no more expensive than a heartbeat function.  A more concrete example:


Suppose Alice, Bob, Carol, and Dan are members of the ring and everyone is ordered alphabetically (Fig. \ref{abcd1}). Alice is quite sure that Carol is her successor.  Alice asks Carol who her predecessor is and Carol says Bob is.  Since Bob is closer than Carol, Alice changes her successor to Bob and notifies him.  

When Bob sees that notification, he can see Alice is closer than whoever his previous predecessor is and sets Alice to be his predecessor.  During the next stabilization cycle, Alice will see that she is still Bob's predecessor and notify him that she's still there (Fig. \ref{fig:abcd2}).

To prevent loss of data due to churn, each node sends a backup of their data to their successor.  Section IV discusses the implementation of the backup process in ChordReduce and expands upon it for backing up Map and Reduce tasks.


\subsection{Extensions of Chord}

The Cooperative File System (CFS) is an anonymous, distributed file sharing system built on top of Chord \cite{CFS}.  In CFS, rather than storing an entire file at a single node, the file is split up into multiple chunks around 10 kilbytes in size.  These chunks are each assigned a hash and stored in nodes corresponding to their hash in the same way that whole files are.  The node that would normally store the whole file instead stores a \emph{key block}, which holds the hash address of the chunks of the file. 

The chunking allows for numerous advantages.  First, it promotes load balancing. Each piece of the overall file would (ideally) be stored in a different node, each with a different backup or backups.  This would prevent any single node from becoming overwhelmed from fulfilling multiple requests for a large file.  It would also prevent retrieval from being bottlenecked by a node with a relatively low bandwidth. Finally, when Chord uses some sort of caching scheme like that described in CFS \cite{CFS}, caching chunks as opposed to the entire file resulted in about 1000 times less storage overhead.  

Mutable files  and  IRM, which is short for Integrated File Replication and Consistancy Maintenence, has nodes keep track of file requests they initiate or forward.  If they find they are frequently forwarding a request for a particular file, they store that file locally until it is no longer requested frequently.  What makes IRM unqiue is that it combines caching with a 

Chunking also opens up the options for implementing additional redundancy such as erasure codes \cite{rizzo1997effective}. With erasure codes, redundant chunks are created but any combination of a particular number of chunks is sufficient to recreate the file.  For example, a file that would normally be split into 10 chunks might be split into 15 encoded chunks.  The retreival of any 10 of those 15 chunks is enough to recreate the file.  Implementing erasure codes would presumably make the network more fault tolerant, but that is an exercise left for future work.


Generally, related files should be kept together; Chord, however, just hashes the filename to find the responsible node and sends it to that location without any thought to organization.  Our solution to this is to use allow the file owner to select first 80 bits of a file's hash, then generating the remaining least signifcant bits by hashing the filename.  It does not matter if a file owner, in some infintesimally small coincidence, chooses the same 80 bit prefix as another file owner, as the purpose is to keep related files together.   
	\chapter{VHash And DGVH}
\label{chapter:dgvh}
DHTs all seek to minimize lookup time for their respective topologies.
This is done by minimizing the number of overlay hops needed for a lookup operation.
This is a good approximation for minimizing the latency of lookups, but does not actually do so, as each hop has a different amount of latency.
Furthermore, a network might need to minimize some arbitrary metric, such as energy consumption.

VHash is a multi-dimensional DHT that minimizes routing over some given metric.
It uses a fast approximation of a Delaunay Triangulation to compute the Voronoi tessellation of a multi-dimensional space.
%Approximated routing tables



Arguably all Distributed Hash Tables (DHTs) are built on the concept of Voronoi tessellations.
In all DHTs, a node is responsible for all points in the overlay to which it is the ``closest'' node.
Nodes are assigned a key as their location in some keyspace, based on the hash of certain attributes.
Normally, this is just the hash of the IP address (and possibly the port) of the node \cite{chord} \cite{kademlia} \cite{can} \cite{pastry}, but other metrics such as geographic location can be used as well \cite{ratnasamy2002ght}.

These DHTs have carefully chosen metric spaces such that these regions are very simple to calculate.
For example, Chord \cite{chord} and similar ring-based DHTs \cite{symphony} utilize a unidirectional, one-dimensional ring as their metric space, such that the region for which a node is responsible is the region between itself and its predecessor.

Using a Voronoi tessellation in a DHT generalizes this design.
Nodes are Voronoi generators at a position based on their hashed keys.
These nodes are responsible for any key that falls within its generated Voronoi region.

Messages get routed along links to neighboring nodes.
This would take $O(n)$ hops in one dimension.
In multiple dimensions, our routing algorithm (Algorithm \ref{alg:lookup}) is extremely similar to the one used in Ratnasamy et al.'s Content Addressable Network (CAN) \cite{can}, which would be $O(n^{\frac{1}{d}})$ hops.


\begin{algorithm}
	\caption{Lookup in a Voronoi-based DHT}
	\label{alg:lookup}
	\begin{algorithmic}[1]
		\State Given node $n$
		\State Given $m$ is a message addressed for $loc$
		\State $potential\_dests \leftarrow n \cup n.short\_peers \cup n.long\_peers$
		\State $c \leftarrow $ node in $ potential\_dests$ with shortest distance to $loc$
		\If{$c$ == $n$}
			\State \Return $n$
		\Else
			\State \Return $c.lookup(loc)$
		\EndIf
	\end{algorithmic}
\end{algorithm}


Efficient solutions, such as Fortune's sweepline algorithm \cite{fortune1987sweepline}, are not usable in spaces with 2 more dimensions.
As far as we can tell, there is no way efficient to generate higher dimension Voronoi tessellations, especially in the distributed Churn-heavy context of a DHT.
Our solution is the Distributed Greedy Voronoi Heuristic.

\section{Distributed Greedy Voronoi Heuristic}
A Voronoi tessellation is the partition of a space into cells or regions along a set of objects $O$, such that all the points in a particular region are closer to one object than any other object.
We refer to the region owned by an object as that object's Voronoi region.
Objects which are used to create the regions are called Voronoi generators.
In network applications that use Voronoi tessellations, nodes in the network act as the Voronoi generators.

The Voronoi tessellation and Delaunay triangulation are dual problems, as an edge between two objects in a Delaunay triangulation exists if and only if those object's Voronoi regions border each other.
This means that solving either problem will yield the solution to both.
An example Voronoi diagram is shown in Figure \ref{voro-ex}.
For additional information, Aurenhammer \cite{voronoi} provides a formal and extremely thorough description of Voronoi tessellations, as well as their applications.


\begin{figure}
	\centering
	\includegraphics[width=0.5\linewidth]{figs/voronoi}
	\caption{An example Voronoi diagram for objects on a 2-dimensional space.  The black lines correspond to the borders of the Voronoi region, while the dashed lines correspond to the edges of the Delaunay Triangulation.}
	\label{voro-ex}
\end{figure}




The Distributed Greedy Voronoi Heuristic (DGVH) is a fast method for nodes to define their individual Voronoi region (Algorithm \ref{alg:dgvh}).
This is done by selecting the nearby nodes that would correspond to the points connected to it by a Delaunay triangulation.
The rationale for this heuristic is that, in the majority of cases, the midpoint between two nodes falls on the common boundary of their Voronoi regions.

%In addition, nodes should only have to compute their own Voronoi region, and possibly estimate those of its neighbors.
%Anything else is a waste of processing power.



\begin{algorithm} % make smaller
	\caption{Distributed Greedy Voronoi Heuristic}
	\label{alg:dgvh}
	\begin{algorithmic}[1]  % the numberis how many lines
		\State Given node $n$ and its list of $candidates$.
		\State Given the minimum $table\_size$
		\State $short\_peers \leftarrow$ empty set that will contain $n$'s one-hop peers
		\State $long\_peers \leftarrow$ empty set that will contain $n$'s two-hop peers
		\State Sort $candidates$ in ascending order by each node's distance to $n$
		\State Remove the first member of $candidates$ and add it to $short\_peers$
		\ForAll{$c$ in $candidates$}
		\State $m$ is the midpoint between $n$ and $c$
		\If{Any node in $short\_peers$ is closer to $m$ than $n$}
		\State Reject $c$ as a peer
		\Else
		\State Remove $c$ from $candidates$
		\State Add $c$ to $short\_peers$
		\EndIf
		\EndFor
		\While{$|short\_peers| < table\_size$ \textbf{and} $|candidates| >0$}
		\State Remove the first entry $c$ from $candidates$
		\State Add $c$ to $short\_peers$
		\EndWhile
		\State Add $candidates$ to the set of $long\_peers$
		\If{$|long\_peers| > table\_size^2$}
		\State $long\_peers \leftarrow$ random subset of $long\_peers$ of size $table\_size^2$
		\EndIf
	\end{algorithmic}
\end{algorithm}


During each cycle, nodes exchange their peer lists with a current neighbor and then recalculate their neighbors.
A node combines their neighbor's peer list with its own to create a list of candidate neighbors.
This combined list is sorted from closest to furthest.
A new peer list is then created starting with the closest candidate.
The node then examines each of the remaining candidates in the sorted list and calculates the midpoint between the node and the candidate.
If any of the nodes in the new peer list are closer to the midpoint than the candidate, the candidate is set aside.
Otherwise the candidate is added to the new peer list.


DGVH never actually solves for the actual polytopes that describe a node's Voronoi region.
This is unnecessary and prohibitively expensive \cite{raynet}.
Rather, once the heuristic has been run, nodes can determine whether a given point would fall in its region.

Nodes do this by calculating the distance of the given point to itself and other nodes it knows about.
The point falls into a particular node's Voronoi region if it is the node to which it has the shortest distance.
This process continues recursively until a node determines that itself to be the closest node to the point.
Thus, a node defines its Voronoi region by keeping a list of the peers that bound it.



\subsection{Algorithm Analysis}

DVGH is very efficient in terms of both space and time.
Suppose a node $n$ is creating its short peer list from $k$ candidates in an overlay network of $N$ nodes.
The candidates must be sorted, which takes $O(k\cdot\lg(k))$ operations.
Node $n$ must then compute the midpoint between itself and each of the $k$ candidates.
Node $n$ then compares distances to the midpoints between itself and all the candidates.
This results in a cost of

\[ k\cdot\lg(k) + k \text{ midpoints}  + k^{2} \text{ distances} \]


Since $k$ is  bounded by $\Theta(\frac{\log N}{\log \log N} )$ \cite{bern1991expected} (the expected maximum degree of a node), we can translate the above to

\[O(\frac{\log^{2} N}{\log^{2} \log N} )\]

In the vast majority of cases, the number of peers is equal to the minimum size of \textit{Short Peers}.
This yields $k=(3d+1)^{2}+3d+1$ in the expected case, where the lower bound and expected complexities are $\Omega(1)$.



\section{Experimental Results}
We evaluated the effectiveness of VHash and DGVH in creating a set of experiments.\footnote{Our results are pulled directly from \cite{dgvh} and \cite{vhash}.}
The first experiment showed how VHash could use DGVH to create a routing mesh.
Our second showed how optimizing for latency yielded better results than optimizing for least hops.

\subsection{Convergence}
Our first experiment examined how DGVH could be used to create a routing overlay and how well it performed in this task.
The simulation demonstrated how DGVH  formed a stable overlay from a chaotic starting topology after a number of cycles.
We compared our results to those in RayNet \cite{raynet}.
The authors of Raynet proposed a random $k$-connected graph would be a challenging initial configuration for showing a DHT relying on a gossip mechanism could converge to a stable topology.

In the initial two cycles of the simulation, each node bootstrapped its short peer list by appending 10 nodes, selected uniformly at random from the entire network.
In each cycle, the nodes gossiped , swapping peer list information.
They then ran DGVH using the new information.
We calculated the hit rate of successful lookups by simulating 2000 lookups from random nodes to random locations, as described in Algorithm \ref{alg:routesim}.
A lookup was considered successful if the network was able to determine which Voronoi region contained a randomly selected point.

Our experimental variables for this simulation were the number of nodes in the DGVH generated overlay and the number of dimensions.
We tested network sizes of 500, 1000, 2000, 5000, and 10000 nodes each in 2, 3, 4, and 5 dimensions.
The hit rate at each cycle is $\frac{hits}{2000}$, where $hits$ are the number of successful lookups.




\begin{algorithm}
	\caption{Routing Simulation Sample}
	\label{alg:routesim}
	\begin{algorithmic}[1]  % the number is how many
		\State $start \leftarrow$ random node
		\State$dest \leftarrow$ random set of coordinates
		\State $ans \leftarrow$ node closest to $dest$
		\If {$ans == start.lookup(dest)$}
		\State increment $hits$
		\EndIf
	\end{algorithmic}
\end{algorithm}

The results of our simulation are shown in Figure \ref{fig:conv}.
Our graphs show that a correct overlay was quickly constructed from a random configuration and that our hit rate reached 90\% by cycle 20, regardless of the number of dimensions.
Lookups consistently approached a hit rate of 100\% by cycle 30.
In comparison, RayNet's routing converged to a perfect hit rate at around cycle 30 to 35 \cite{raynet}.
As the network size and number of dimensions each increase, convergence slows, but not to a significant degree.

\begin{figure*}
	\centering
	\begin{tabular}{cc}

		\begin{subfigure}{0.5\columnwidth}
			\includegraphics[width=\linewidth]{figs/conv_d2}
			\caption{This plot shows the accuracy rate of lookups on a 2-dimensional network as it self-organizes.}
			\label{fig:conv2}
		\end{subfigure} &

		\begin{subfigure}{0.5\columnwidth}
			\includegraphics[width=\linewidth]{figs/conv_d3}
			\caption{This plot shows the accuracy rate of lookups on a 3-dimensional network as it self-organizes.}
			\label{fig:conv3}
		\end{subfigure} \\

		\begin{subfigure}{0.5\columnwidth}
			\includegraphics[width=\linewidth]{figs/conv_d4}
			\caption{This plot shows the accuracy rate of lookups on a 4-dimensional network as it self-organizes.}
			\label{fig:conv4}
		\end{subfigure} &


		\begin{subfigure}{0.5\columnwidth}
			\includegraphics[width=\linewidth]{figs/conv_d5}
			\caption{This plot shows the accuracy rate of lookups on a 5-dimensional network as it self-organizes.}
			\label{fig:conv5}
		\end{subfigure}

	\end{tabular}
	\caption{These figures show that, starting from a randomized network, DGVH forms a stable and consistent network topology.
		The Y axis shows the success rate of lookups and the X axis show the number of gossips that have occurred.
		Each point shows the fraction of 2000 lookups that successfully found the correct destination.}
	
	\label{fig:conv}

\end{figure*}

\subsection{Latency Distribution Test}
The goal of our second set of experiments was to demonstrate VHash's ability to optimize a selected network metric: latency in this case.
In our simulation, we used the number of hops on the underlying network as an approximation of latency.
We compared VHash's performance to Chord \cite{chord}.
As we discussed in Chapter \ref{chapter:background} Chord is a well established DHT with an $O(\log(n))$ sized routing table and $O(\log(n))$ lookup time measured in overlay hops.

Instead of using the number of hops on the overlay network as our metric, we are concerned with the actual latency lookups experience traveling through the \emph{underlay} network, the network upon which the overlay is built.
Overlay hops are used in most DHT evaluations as the primary measure of latency.
It is the best approach available when there are no means of evaluating the characteristics of the underlying network.
VHash is designed with a capability to exploit the characteristics of the underlying network.
With most realistic network sizes and structures, there is substantial room for latency reduction in DHTs.

For this experiment, we constructed a scale free network with 10000 nodes placed at random (which has an approximate diameter of 3 hops) as an underlay network \cite{cohen2000resilience} \cite{pastor2001epidemic} \cite{hagberg2004}.
We chose to use a scale-free network as the underlay, since  scale free networks model the Internet's topology \cite{cohen2000resilience} \cite{pastor2001epidemic}.
We then chose a random subset of nodes to be members of the overlay network.
Our next step was to measure the distance in underlay hops between 10000 random source-destination pairs in the overlay.
VHash generated an embedding of the latency graph utilizing a distributed force directed model, with the latency function defined as the number of underlay hops between it and its peers.

Our simulation created 100, 500, and 1000 node overlays for both VHash and Chord.
We used 4 dimensions in VHash and a standard 160 bit identifier for Chord.




\begin{figure}

\begin{subfigure}{\columnwidth}
\centering
	\includegraphics[width=0.5\linewidth]{figs/hist_100}
	\caption{Frequency of path lengths on Chord and VHash in a 100 node overlay.}
	\label{fig:hist100}
\end{subfigure}

\begin{subfigure}{\columnwidth}
	\centering
	\includegraphics[width=0.5\linewidth]{figs/hist_500}
	\caption{Frequency of path lengths on Chord and VHash in a 500 node overlay.}
	\label{fig:hist500}
\end{subfigure}

\begin{subfigure}{\columnwidth}
	\centering
	\includegraphics[width=0.5\linewidth]{figs/hist_1000}
	\caption{Frequency of path lengths on Chord and VHash in a 1000 node overlay.}
	\label{fig:hist1000}
\end{subfigure}

\caption{Figures \ref{fig:hist100}, \ref{fig:hist500}, and \ref{fig:hist1000} show the difference in the performance of Chord and VHash for 10,000 routing samples on a 10,000 node underlay network for differently sized overlays.
The Y axis shows the observed frequencies and the X axis shows the number of hops traversed on the underlay network.
VHash consistently requires fewer hops for routing than Chord.}
\label{fig:hist}

\end{figure}




Figure \ref{fig:hist} shows the distribution of path lengths measured in underlay hops in both Chord and VHash.
VHash significantly outperformed Chord and considerably reduced the underlay path lengths in three network sizes.

We also sampled the lookup length measured in overlay hops for a 1000 sized Chord and VHash network.
As seen in Figure \ref{fig:histover}, the paths measured in overlay for VHash were significantly shorter than those in Chord.
In comparing the overlay and underlay hops, we find that for each overlay hop in Chord, the lookup must travel 2.719 underlay hops on average; in VHash, lookups must travel 2.291 underlay hops on average for every overlay hop traversed.

Recall that this work is based on scale free networks, where latency improvements are difficult.
An improvement of 0.4 hops over a diameter of 3 hops is significant.
VHash has on average less overlay hops per lookup than Chord, and for each of these overlay hops we consistently traverse more efficiently across the underlay network.
\begin{figure}
	\centering
	\includegraphics[width=0.5\linewidth]{figs/hist_overlay_4d}
	\caption{Comparison of Chord and VHash in terms of overlay hops.  Each overlay has 1000 nodes.  The Y axis denotes the observed frequencies of overlay hops and the X axis corresponds to the path lengths in overlay hops.}
	\label{fig:histover}
\end{figure}




\section{Remarks}

Voronoi tessellations have a wide potential for applications in ad-hoc networks, massively multiplayer games, P2P, and distributed networks.
However, centralized algorithms for Voronoi tessellation and Delaunay triangulation are not applicable to decentralized systems.
In addition, solving Voronoi tessellations in more than 2 dimensions is computationally expensive.

We created a distributed heuristic for Voronoi tessellations in an arbitrary number of dimensions.
Our heuristic is fast and scalable, with a expected memory cost of $(3d+1)^{2}+3d+1$ and expected maximum runtime of O$(\frac{\log^{2} N}{\log^{2} \log N} )$.

We ran two sets of experiments to demonstrate VHash's effectiveness.
Our first set of experiments demonstrated that our heuristic is reasonably accurate  and our second set demonstrates that reasonably accurate is sufficient to build a P2P network which can route accurately.
Our second experiment showed that VHash  could significantly reduced the latency in Distributed Hash Tables.

	\include{urdht}
	\include{d3dns}
	\chapter{Analysis of The Sybil Attack}
\label{chapter:sybil}
	\chapter{Autonomous Load Balancing}
\label{chapter:auto-balance}




%http://michiel.buddingh.eu/distribution-of-hash-values

\section{Introduction}
Distributed Hash Tables rely on cryptographic hash functions to generate identifiers for both nodes and data.
Ideally, inputing random numbers into a cryptographic hash function should produce a uniformly distributed output.
However, this is impossible in practice \cite{hash-outputs} \cite{thomsen2005cryptographic}.


\begin{table}
	\centering
	\caption{The median distribution of tasks among nodes.  We can see the standard deviation is a particularly useless measurement in this scenario.}
	\begin{tabular}{r r r r}
		Nodes & Tasks & Median Workload & $\sigma$ \\ \hline
		1000  & 100000 & 69.410  & 137.27  \\
		
		1000 & 500000 & 346.570  &  499.169 \\
		1000 &1000000 & 692.300  &  996.982 \\
		
		5000 & 100000  & 13.810 & 20.477 \\ 
		5000 & 500000  & 69.280 & 100.344 \\ 
		5000 & 1000000 &138.360 & 200.564 \\ 
		
		10000 & 100000 & 7.000   &  10.492 \\
		10000 & 500000 & 34.550  &   50.366 \\
		10000 & 1000000& 69.180  &  100.319 \\
	\end{tabular}
	\label{tab:medianLoads}
\end{table}




In practice, that means given any DHT with files and nodes, there will be an inherent imbalance in the network.
Some nodes will end up with a lion's share of the keys, while other will have few responsibilities (Table \ref{tab:medianLoads}).

This makes it especially disheartening to try and ensure as even a load as possible.
We cannot rely on a centralized strategy to fix this imbalance, since that would violate the principles and objects behind creating a fully decentralized and distributed system.

Therefore, if we want to create strategies to act against the inequity of the load distribution, we need a strategy that individual nodes can act upon autonomously.
These strategies need to make decisions that a node can make at a local level, using only information about their own load and the topology they can see.


\subsection{Motivation}
The primary motivation for us is creating a new viable type of platform for distributed computing.
Most distributed computing paradigms, such as Hadoop \cite{hadoop}, assume that the computations occur in a centralized environments.
One enormous benefit is a centralized system has much greater control in ensuring load-balancing.

However, in an increasingly global world where computation is king and the Internet is increasingly an integral part of everyday life, single points of failure failures quickly become more and more risky.
Users expect their apps to work regardless of any power outage affecting an entire region.
Customers expect their services to still be provided regardless of any.
The next big quake affecting the the San Andreas fault line is a matter of when, not if.
Thus, centralized systems with single points of failure become a riskier option and decentralized, distributed systems the safer choice.


Our previous work in ChordReduce \cite{chordreduce} focused on creating a system

\subsection{Objectives}


\section{How Work Is Distributed in DHTs: A Visual Guide}
In this section, we display graphs to give a visual representation of how work is distributed in a Chord \cite{chord} network

\section{The Simulation}

%We simulate an UrDHT Voronoi based-network in multiple dimensions. \footnote{UrDHT in one dimension is a Chord ring with the definition of responsibility changed to a node being responsible to all data closest to it. A 2-dimensional network will emulate the performance of CAN.}

We assume that the network starts our experiments stable and the data necessary already present on the nodes and backed-up.
The following analysis and simulation relies on an important assumption about DHT behavior often assumed but not necessarily implemented.

We assume that nodes are active and aggressive in creating and monitoring the backups and the data they are responsible for.
Specifically, we will assume it takes  $T_{detect}$ time for a node to detect a change in their responsibility or to detect a new node to hand a backup to and that this check is performed regularly. 
We have demonstrated the effectiveness and viability of implementing an active backup strategy in other work \cite{chordreduce} \cite{brendanBackup} \cite{urdht}.


Another assumption is that nodes do not have control in choosing their IDs from the range of hash values.

Smaller chunking results in more files spread throughout the  network and a greater chance of the data being evenly spread across the network 

The chances of a critical failure happening within a time interval $ T $ is the chances of some chain or cluster of nodes responsible for a single record dying within $ T $:

$$r^{s}$$

Where $ r $ is the failure rate over that time interval and $s$ is the number nodes storing that record, either as a primary system, or a backup.
Incidentally, this time interval $T = T_{detect} + T_{transfer} $



\subsection{The Parameters}

\subsubsection{Constants}

\begin{description}
	\item [Time Unit] In a simulation, normal measurements of time such as a second are arbitrary, so we be using the abstract \textit{tick} to measure time.  
	If we want to be more concrete, a tick is the amount of time it takes a node to complete one task per sybil and perform the appropriate maintanence.\footnote{The shortest unit of time in the multiverse is the New York Second, defined as the period of time between the traffic lights turning green and the cab behind you honking.\\-- Terry Pratchett}
	\item [Maintence] We assume the reactive, aggressive backups works.
	\item [Hash Functions] We will be using SHA-1 \cite{sha1}, a 160-bit hash function.  
	Keys will be drawn randomly from this hash function.
\end{description}

\subsubsection{Experimental Variables}
\begin{description}
	\item [Churn] Measured in ticks, this can be self induced or a result of actual turbulence in the network.
	Like most analyses of churn \cite{marozzo2012p2p}, we assume churn is constant throughout the experiment and that the joining and leaving rate are equal.
	\item [Network Size]  How many nodes start in the network.  
		We assume that this can grow during the experiment, either via churn or by creating sybils.
	\item [Size of the job] The size of the job, in tasks.
		This number is typically orders of maginitude greater than the network size.
\end{description}

We also considered using a variable noted as the \texttt{AdaptationRate}, which was how long it would take
Preliminary experiments \texttt{AdaptationRate} 

\subsubsection{Outputs}



\section{Strategies}

For our analysis, we examined four different strategies for nodes to perform autonomous load bal



\subsection{Induced Churn}
In this experiment, we rely solely on churn to perform load balancing.
This churn can either be a product of normal netowrk activity or self-induced.
Self-induced churn means that each node generates a random floating point number between 0 and 1.
If the number is $\leq churnRate$, the node shuts down and leaves the network.

Similarly, we have a pool of waiting nodes the same size as the network.
When they generate an appropriate random number, they join the network.

In our network model, nodes actively back up their data and tasks to nodes as they enter and leave the network.
While this model is rarely implemented for DHTs, we have implemented it in ChordReduce\cite{chordreduce} and UrDHT\cite{urdht} and demonstrated that the network is capable from recovering from quite catastrophic failures.

In this model, a node suddenly dying is of minimal impact to the network.
This is because a node's successor will quickly detect the loss of the node and know to be responsible for the node's work.

Conversely, a node joining in this model can be a potential boon to the network by joining a portion of the network with a lot of tasks.



We assume that nodes enter and leave the network at the same rate.


\subsection{Random Sybil Injection}
Our second series of experiments focused on nodes with low amounts of work performing a controlled and strategic Sybil attack \cite{sybil} on the network.
In this experiment, each once each node's workload was at or below a certain threshold, the node would attempt to acquire more work by creating virtual Sybil nodes at random addresses.




No benefit was shown by increasing maxSybils beyond 10, so we stopped increasing it there.


\subsection{Neighbor Injection}

\subsection{Invitation}

In invitation, churn losses can be greatly detrimental

\section{Results Of Experiments}

	\include{conclusion}
	
	\bibliography{notes,dht,mapreduce,voronoi,dns,botnets,bitcoin,mine}
	\bibliographystyle{plain}
\end{document}
