\chapter{Autonomous Load Balancing in a Distributed Hash Table}
\label{chapter:auto-balance}




%http://michiel.buddingh.eu/distribution-of-hash-values

\section{Introduction}
Distributed Hash Tables rely on cryptographic hash functions to generate identifiers for both nodes and data.
Ideally, inputing random numbers into a cryptographic hash function should produce a uniformly distributed output.
However, this is impossible in practice \cite{hash-outputs} \cite{thomsen2005cryptographic}.


\begin{table}
	\centering
	\caption{The median distribution of tasks among nodes.  We can see the standard deviation is a particularly useless measurement in this scenario.}
	\begin{tabular}{r r r r}
		Nodes & Tasks & Median Workload & $\sigma$ \\ \hline
		1000 & 100000 & 69.410   &  137.27  \\
		1000 & 500000 & 346.570  &  499.169 \\
		1000 &1000000 & 692.300  &  996.982 \\
		
		5000 & 100000  & 13.810 & 20.477 \\ 
		5000 & 500000  & 69.280 & 100.344 \\ 
		5000 & 1000000 &138.360 & 200.564 \\ 
		
		10000 & 100000 & 7.000   &  10.492 \\
		10000 & 500000 & 34.550  &   50.366 \\
		10000 & 1000000& 69.180  &  100.319 \\
	\end{tabular}
	\label{tab:medianLoads}
\end{table}




In practice, that means given any DHT with files and nodes, there will be an inherent imbalance in the network.
Some nodes will end up with a lion's share of the keys, while other will have few responsibilities (Table \ref{tab:medianLoads}).

This makes it especially disheartening to try and ensure as even a load as possible.
We cannot rely on a centralized strategy to fix this imbalance, since that would violate the principles and objects behind creating a fully decentralized and distributed system.

Therefore, if we want to create strategies to act against the inequity of the load distribution, we need a strategy that individual nodes can act upon autonomously.
These strategies need to make decisions that a node can make at a local level, using only information about their own load and the topology they can see.


\subsection{Motivation}
The primary motivation for us is creating a new viable type of platform for distributed computing.
Most distributed computing paradigms, such as Hadoop \cite{hadoop}, assume that the computations occur in a centralized environments.
One enormous benefit is a centralized system has much greater control in ensuring load-balancing.

However, in an increasingly global world where computation is king and the Internet is increasingly an integral part of everyday life, single points of failure failures quickly become more and more risky.
Users expect their apps to work regardless of any power outage affecting an entire region.
Customers expect their services to still be provided regardless of any.
The next big quake affecting the the San Andreas fault line is a matter of when, not if.
Thus, centralized systems with single points of failure become a riskier option and decentralized, distributed systems the safer choice.


Our previous work in ChordReduce \cite{chordreduce} focused on creating a decentralized distributed computing framework based off of the Chord Distributed Hash Table (DHT) and MapReduce.
ChordReduce can be thought of a more generalized implementation of the concepts of MapReduce.
One of the advantages of ChordReduce can be used in either a traditional datacenter or P2P environment.\footnote{The other one being that new nodes could join during runtime and receive work from nodes doing computations.}
Chord (and all DHTs) have the qualities we desire for distributed computing: scalability, fault tolerance, and load-balancing.

Fault tolerance is of particular importance to DHTs, since their primary use case is P2P file-sharing, such as BitTorrent \cite{bittorrent}.
These systems experience high levels of churn-- disruptions to the network topology as a result of nodes entering and leaving the network.
ChordReduce had to have the same level of robustness against churn that Chord did, if not better.

During our experiments with ChordReduce, we found that high levels of churn actually made our computations run \textit{faster}.
We hypothesized that the churn was effectively load-balancing the network.

\subsection{Objectives}
This paper serves to prove our hypothesis that churn can load balance a Distributed Hash Table.
We also set out to show that we can use this in a highly controlled manner to greater effect.
We present 3 strategies nodes can use to redistribute the workload in the network that do no require a centralized organizer.

We also want to show how distributed computing can be performed in a heterogeneous environment.

\section{Previous Work}


ChordReduce \cite{chordreduce} is designed as a more abstract framework for MapReduce, able to run on any arbitrary distributed configuration.
ChordReduce leverages the features of distributed hash tables to handle distributed file storage, fault tolerance, and lookup.
We designed ChordReduce to ensure that no single node is a point of failure and that there is no need for any node to coordinate the efforts of other nodes during processing.

\section{How Work Is Distributed in DHTs: A Visual Guide}
In this section, we display graphs to give a visual representation of how work is distributed in a Chord \cite{chord} network

\section{Simulation}

%We simulate an UrDHT Voronoi based-network in multiple dimensions. \footnote{UrDHT in one dimension is a Chord ring with the definition of responsibility changed to a node being responsible to all data closest to it. A 2-dimensional network will emulate the performance of CAN.}

We assume that the network starts our experiments stable and the data necessary already present on the nodes and backed-up.
The following analysis and simulation relies on an important assumption about DHT behavior often assumed but not necessarily implemented.

We assume that nodes are active and aggressive in creating and monitoring the backups and the data they are responsible for.
Specifically, we will assume it takes  $T_{detect}$ time for a node to detect a change in their responsibility or to detect a new node to hand a backup to and that this check is performed regularly. 
We have demonstrated the effectiveness and viability of implementing an active backup strategy in other work \cite{chordreduce} \cite{urdht}.


Another assumption is that nodes do not have control in choosing their IDs from the range of hash values.

Smaller chunking results in more files spread throughout the  network and a greater chance of the data being evenly spread across the network 

The chances of a critical failure happening within a time interval $ T $ is the chances of some chain or cluster of nodes responsible for a single record dying within $ T $:

$$r^{s}$$

Where $ r $ is the failure rate over that time interval and $s$ is the number nodes storing that record, either as a primary system, or a backup.
Incidentally, this time interval $T = T_{detect} + T_{transfer} $



\subsection{The Parameters}

Our simulations relied on a great number of parameters and variables.
We present each of them below.

\subsubsection{Constants and Terminology}

\begin{description}
	\item [Tick] In a simulation, normal measurements of time such as a second are arbitrary, so we be using the abstract \textit{tick} to measure time.  
	We consider the tick the amount of time it takes a node to complete one task per sybil and perform the appropriate maintanence.\footnote{If we need to be more concrete, define a tick as a New York Second ``defined as the period of time between the traffic lights turning green and the cab behind you honking.''\\-- Terry Pratchett}
	\item [Maintence] We assume nodes use the active, aggressive strategy from ChordReduce and  UrDHT \cite{chordreduce} \cite{urdht}.
	Every maintenance cycle, each node checks and updates its list of neighbors (successors and predecessors in Chord) and responds appropriately . 
	We assume that a tick
	\item [Task] A distributed computing job is defined in 
	\item [Hash Functions] We will be using SHA-1 \cite{sha1}, a 160-bit hash function.  
	Keys for  will be drawn randomly from this hash function.
\end{description}

\subsubsection{Experimental Variables}
\begin{description}
	\item [Strategy] Which of the strategies (discussed in Section \ref{sec:strategies}) the trial uses for load balancing.
	\item [Churn] Measured in ticks, this can be self induced or a result of actual turbulence in the network.
	Like most analyses of churn \cite{marozzo2012p2p}, we assume churn is constant throughout the experiment and that the joining and leaving rate are equal.
	\item [Network Size]  How many nodes start in the network.  
		We assume that this can grow during the experiment, either via churn or by creating Sybils.
	\item [Size of the job] The size of the job, in tasks.
		This number is typically a few orders of magnitude greater than the network size.
\end{description}

We also considered using a variable noted as the \texttt{AdaptationRate}, which was how long it would take
Preliminary experiments \texttt{AdaptationRate} 

\subsubsection{Outputs}



\section{Strategies}
\label{sec:strategies}

For our analysis, we examined four different strategies for nodes to perform autonomous load balancing.
We first show the effects of churn on the speed of a distributed computation.
We then look a three different strategies for in which nodes take a more tactical approach for creating churn and affecting the runtime.

Specifically, nodes perform a limited and controlled Sybil attack \cite{sybil} on their own network in an effort to acquire work with their virtual nodes.
Our strategies dictate when and where these Sybil nodes are created.

We discuss the effectiveness of this strategy in Section \ref{sec:autonomous-results}.



\subsection{Induced Churn}
Our first strategy, \textit{Induced Churn}, relies solely on churn to perform load balancing.
This churn can either be a product of normal network activity or self-induced.\footnote{All distributed systems experience churn, even if only as hardware failures.}
By self-induced churn, we mean that each node generates a random floating point number between 0 and 1.
If the number is $\leq churnRate$, the node shuts down and leaves the network.

Similarly, we have a pool of waiting nodes the same size as the network.
When they generate an appropriate random number, they join the network.
We assume that nodes enter and leave the network at the same rate.


As we have previously discussed, nodes in our network model actively back up their data and tasks to the a number of successors in case of failure.
In addition, when a node joins, it acquires all the work it is responsible for.
While this model is rarely implemented for DHTs, it is discussed \cite{kademlia} and often assumed to be the way DHTs operate. 
We have implemented it in ChordReduce\cite{chordreduce} and UrDHT\cite{urdht} and demonstrated that the network is capable from recovering from quite catastrophic failures and handling ludicrous amounts of churn.

The consequences of this are that a node suddenly dying is of minimal impact to the network.
This is because a node's successor will quickly detect the loss of the node and know to be responsible for the node's work.
Conversely, a node joining in this model can be a potential boon to the network by joining a portion of the network with a lot of tasks and immediately acquiring work.

This strategy acts as a baseline with which to compare the other strategies, as it is no more than a overcomplicated way of turning machines off and on again. 


\subsection{Random Sybil Injection}
Our second strategy we dubbed \textit{Random Injection}.
In this strategy, once a node's workload was at or below the \texttt{sybilThreshold}, the node would attempt to acquire more work by creating a Sybil node at a random address.

A node checks its threshold and decides whether or not to make a Sybil every 5 ticks.
A node cam also have multiple Sybils, up to \texttt{maxSybils} in a homogeneous network or the node's \texttt{strength} in a heterogeneous network.\footnote{No benefit was shown by increasing \texttt{maxSybils} beyond 10.}
If a node has at least one Sybil, but no work, it has its Sybils quit the network.
We have the decision to make a Sybil occur every 5 ticks and limit each node to creating a single Sybil a check to avoid overwhelming the network.

As we discuss in Section \ref{sec:autonomous-results}, this strategy is surprisingly effective and comes close to ideal runtimes.


\subsection{Neighbor Injection}
\textit{Neighbor Injection} also creates Sybils, but in this case


\subsection{Invitation}

In invitation, churn losses can be greatly detrimental

\section{Results Of Experiments}
\label{sec:autonomous-results}

