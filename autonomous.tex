\chapter{Autonomous Load Balancing}
\label{chapter:auto-balance}




%http://michiel.buddingh.eu/distribution-of-hash-values

\section{Introduction}
Distributed Hash Tables rely on cryptographic hash functions to generate identifiers for both nodes and data.
Ideally, inputing random numbers into a cryptographic hash function should produce a uniformly distributed output.
However, this is impossible in practice \cite{hash-outputs} \cite{thomsen2005cryptographic}.


\begin{table}
	\centering
	\caption{The median distribution of tasks among nodes.  We can see the standard deviation is a particularly useless measurement in this scenario.}
	\begin{tabular}{r r r r}
		Nodes & Tasks & Median Workload & $\sigma$ \\ \hline
		1000  & 100000 & 69.410  & 137.27  \\
		
		1000 & 500000 & 346.570  &  499.169 \\
		1000 &1000000 & 692.300  &  996.982 \\
		
		5000 & 100000  & 13.810 & 20.477 \\ 
		5000 & 500000  & 69.280 & 100.344 \\ 
		5000 & 1000000 &138.360 & 200.564 \\ 
		
		10000 & 100000 & 7.000   &  10.492 \\
		10000 & 500000 & 34.550  &   50.366 \\
		10000 & 1000000& 69.180  &  100.319 \\
	\end{tabular}
	\label{tab:medianLoads}
\end{table}




In practice, that means given any DHT with files and nodes, there will be an inherent imbalance in the network.
Some nodes will end up with a lion's share of the keys, while other will have few responsibilities (Table \ref{tab:medianLoads}).

This makes it especially disheartening to try and ensure as even a load as possible.
We cannot rely on a centralized strategy to fix this imbalance, since that would violate the principles and objects behind creating a fully decentralized and distributed system.

Therefore, if we want to create strategies to act against the inequity of the load distribution, we need a strategy that individual nodes can act upon autonomously.
These strategies need to make decisions that a node can make at a local level, using only information about their own load and the topology they can see.


\subsection{Motivation}
The primary motivation for us is creating a new viable type of platform for distributed computing.
Most distributed computing paradigms, such as Hadoop \cite{hadoop}, assume that the computations occur in a centralized environments.
One enormous benefit is a centralized system has much greater control in ensuring load-balancing.

However, in an increasingly global world where computation is king and the Internet is increasingly an integral part of everyday life, single points of failure failures quickly become more and more risky.
Users expect their apps to work regardless of any power outage affecting an entire region.
Customers expect their services to still be provided regardless of any.
The next big quake affecting the the San Andreas fault line is a matter of when, not if.
Thus, centralized systems with single points of failure become a riskier option and decentralized, distributed systems the safer choice.


Our previous work in ChordReduce \cite{chordreduce} focused on creating a system

\subsection{Objectives}


\section{How Work Is Distributed in DHTs: A Visual Guide}
In this section, we display graphs to give a visual representation of how work is distributed in a Chord \cite{chord} network

\section{The Simulation}

%We simulate an UrDHT Voronoi based-network in multiple dimensions. \footnote{UrDHT in one dimension is a Chord ring with the definition of responsibility changed to a node being responsible to all data closest to it. A 2-dimensional network will emulate the performance of CAN.}

We assume that the network starts our experiments stable and the data necessary already present on the nodes and backed-up.
The following analysis and simulation relies on an important assumption about DHT behavior often assumed but not necessarily implemented.

We assume that nodes are active and aggressive in creating and monitoring the backups and the data they are responsible for.
Specifically, we will assume it takes  $T_{detect}$ time for a node to detect a change in their responsibility or to detect a new node to hand a backup to and that this check is performed regularly. 
We have demonstrated the effectiveness and viability of implementing an active backup strategy in other work \cite{chordreduce} \cite{brendanBackup} \cite{urdht}.


Another assumption is that nodes do not have control in choosing their IDs from the range of hash values.

Smaller chunking results in more files spread throughout the  network and a greater chance of the data being evenly spread across the network 

The chances of a critical failure happening within a time interval $ T $ is the chances of some chain or cluster of nodes responsible for a single record dying within $ T $:

$$r^{s}$$

Where $ r $ is the failure rate over that time interval and $s$ is the number nodes storing that record, either as a primary system, or a backup.
Incidentally, this time interval $T = T_{detect} + T_{transfer} $



\subsection{The Parameters}

\subsubsection{Constants}

\begin{description}
	\item [Time Unit] In a simulation, normal measurements of time such as a second are arbitrary, so we be using the abstract \textit{tick} to measure time.  
	If we want to be more concrete, a tick is the amount of time it takes a node to complete one task per sybil and perform the appropriate maintanence.\footnote{The shortest unit of time in the multiverse is the New York Second, defined as the period of time between the traffic lights turning green and the cab behind you honking.\\-- Terry Pratchett}
	\item [Maintence] We assume the reactive, aggressive backups works.
	\item [Hash Functions] We will be using SHA-1 \cite{sha1}, a 160-bit hash function.  
	Keys will be drawn randomly from this hash function.
\end{description}

\subsubsection{Experimental Variables}
\begin{description}
	\item [Churn] Measured in ticks, this can be self induced or a result of actual turbulence in the network.
	Like most analyses of churn \cite{marozzo2012p2p}, we assume churn is constant throughout the experiment and that the joining and leaving rate are equal.
	\item [Network Size]  How many nodes start in the network.  
		We assume that this can grow during the experiment, either via churn or by creating sybils.
	\item [Size of the job] The size of the job, in tasks.
		This number is typically orders of maginitude greater than the network size.
\end{description}

We also considered using a variable noted as the \texttt{AdaptationRate}, which was how long it would take
Preliminary experiments \texttt{AdaptationRate} 

\subsubsection{Outputs}



\section{Strategies}

For our analysis, we examined four different strategies for nodes to perform autonomous load bal



\subsection{Induced Churn}
In this experiment, we rely solely on churn to perform load balancing.
This churn can either be a product of normal netowrk activity or self-induced.
Self-induced churn means that each node generates a random floating point number between 0 and 1.
If the number is $\leq churnRate$, the node shuts down and leaves the network.

Similarly, we have a pool of waiting nodes the same size as the network.
When they generate an appropriate random number, they join the network.

In our network model, nodes actively back up their data and tasks to nodes as they enter and leave the network.
While this model is rarely implemented for DHTs, we have implemented it in ChordReduce\cite{chordreduce} and UrDHT\cite{urdht} and demonstrated that the network is capable from recovering from quite catastrophic failures.

In this model, a node suddenly dying is of minimal impact to the network.
This is because a node's successor will quickly detect the loss of the node and know to be responsible for the node's work.

Conversely, a node joining in this model can be a potential boon to the network by joining a portion of the network with a lot of tasks.



We assume that nodes enter and leave the network at the same rate.


\subsection{Random Sybil Injection}
Our second series of experiments focused on nodes with low amounts of work performing a controlled and strategic Sybil attack \cite{sybil} on the network.
In this experiment, each once each node's workload was at or below a certain threshold, the node would attempt to acquire more work by creating virtual Sybil nodes at random addresses.




No benefit was shown by increasing maxSybils beyond 10, so we stopped increasing it there.


\subsection{Neighbor Injection}

\subsection{Invitation}

\section{Results Of Experiments}
