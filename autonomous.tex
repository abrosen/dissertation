\chapter{Autonomous Load Balancing}
\label{chapter:auto-balance}



The following analysis and simulation relies on an important assumption about DHT behavior often assumed but not necessarily implemented.
We assume that nodes are active and aggressive in creating and monitoring the backups and the data they are responsible for.
Specifically, we will assume  it takes  $T_{detect}$ time for a node to detect a change in their responsibility or to detect a new node to hand a backup to and that this check is performed regularly.
Another assumption is that nodes do note have control in choosing their IDs from the range of hash values.

Smaller chunking results in more files spread throughout the  network and a greater chance of the data being evenly spread across the network 

The chances of a critical failure happening within a time interval $ T $ is the chances of some chain or cluster of nodes responsible for a single record dying within $ T $:

$$r^{s}$$

Where $ r $ is the failure rate over that time interval and $s$ is the number nodes storing that record, either as a primary system, or a backup.
Incidentally, this time interval $T = T_{detect} + T_{transfer} $
