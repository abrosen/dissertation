\chapter{Conclusion}
\label{chapter:conclusion}
Distributed Hash Tables (DHTs) are extremely powerful frameworks for distributed applications that are based off the simple and powerful hash tables.
Because DHTs were designed with P2P applications in mind, DHTs are scalable, fault-tolerant, and load-balancing.
These are exactly the qualities needed in a distributed computing framework.

We created a new DHT called VHash, which was based off the relationship we discovered between the DHTs and Voronoi tessellation.
VHash is a DHT that operates over a multidimensional space, which allows the embedding of arbitrary metrics into this space.
We showed that we can optimize latency in VHash to obtain faster lookup speeds than a traditional DHT, such as Chord.
The key to VHash is our Distributed Greedy Voronoi Heuristic (DGVH).
DGVH is a sufficiently accurate and fast approximation of Delaunay triangulations.
Aside from its application in VHash, DGVH's applications extend to other areas, such as wireless sensor networks.

We have also shown in the previous chapters that we were able to create a prototype distributed computing application \cite{chordreduce} based on the Chord DHT \cite{chord}.
ChordReduce, as we named it, demonstrated how MapReduce could be performed on the Chord distributed hash table.
As a DHT, ChordReduce is completely decentralized and fault-tolerant, able to handle  nodes entering and leaving the network during churn.
We demonstrated that ChordReduce can efficiently distribute a problem whilst undergoing significant churn and achieve a significant speedup.

During our experiments with ChordReduce, we found an anomaly in which a computation undergoing a significantly high level of churn finished twice as fast than when no churn was involved.
This implied to us that there the churn was effectively shuffling around the nodes such that nodes with no work were taking work from nodes with large amounts of work.
We want to use this implication  develop a more intelligent autonomous load-balancing mechanism.
Such a mechanism would allow underworked nodes to steal work from overworked nodes in the network.

Part of autonomous load-balancing will involve exploiting heterogeneity in the network.
We can do this by having more powerful nodes take a proportionally higher amount of work.
This involves a process we dubbed \textit{mashing}, which we originally used to analyze the Sybil attack on DHTs \cite{sybil-analysis}.


Based on the work we have completed, we proposed creating a framework, called UrDHT, and use it to create a distributed computing. 
As the name implies, UrDHT is meant to be the prototypical DHT, from which we can derive all other DHTs.
This framework would make it easy for developers to create not only new DHTs, but new distributed and P2P applications.
The application that we plan on creating with UrDHT is a Distributed Computing framework based on ChordReduce.

Our new framework will be able to handle more than just MapReduce problems and incorporate an autonomous load-balancing mechanism,
Developers could use our framework to effortlessly organize a disparate set of nodes into a functional distributed computing system and run their own applications.
Our framework could be used in numerous contexts, be it P2P or a data center.